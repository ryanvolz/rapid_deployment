% EndHDF5Spec.tex is a LaTeX file that defines the structure
% of the output of the final processing stage of the 
% MIDAS-M beacon receiver.
%
% $Id: EndHDF5Spec.tex 141 2007-08-10 17:56:39Z hguturu $

\documentclass[12pt]{article}
\begin{document}
\begin{center}
\Large{Beacon Receiver End HDF5 Output File}\\
\large{(Fourth File, Stage D)}\\
\large{Damian Ancukiewicz, Harendra Guturu}\\
\large{Aug 05, 2007}
\end{center}

\begin{small}
\paragraph{Description}
This document specifies the format of the file that the receiver outputs after a satellite flyby once all stages of processing are done. This file is used by a loader program to import some of its elements into the Madrigal database. Additionally, it is stored so that it can be accessed by a separate user interface for scientific or debugging purposes.
\paragraph{Definitions}
These dimensions are determined by the data and metadata contained inside the HDF5 file and are not hard-coded.\\

\begin{tabular}{l c p{10cm}}
$N_{freq}$ &-& Number of distinct frequencies (ex.~VHF, UHF, L-band) that are processed from the same satellite \\
$N_{samp}$ &-& Number of complex voltage samples present\\
$N_{out}$ &-& Number of output frequency points and their corresponding snr, lock, deviation and error points present\\
$N_{s4}$ &-& Number of $S_4$ points present\\
$N_{\sigma \phi}$ &-& Number of $\sigma _\phi$ points present\\
$N_{tec}$ &-& Number of TEC points present\\
\end{tabular}

\paragraph{Structure}
The file is divided into directories, which contain arrays of various types and dimensions. 

\paragraph{Metadata}- contains all necessary metadata for the pass.\\
\begin{scriptsize}
\begin{tabular}{|l|l|l|c|p{7cm}|}
\hline
\textbf{Name} & \textbf{Size} & \textbf{Type} & \textbf{Units} & \textbf{Description} \\
\hline

Fsat & $(N_{freq})$ & Float64 & Hz & Defined frequencies of each beacon (disregarding Doppler shift)\\
Bandwidth & ($N_{freq}$) & Float64 & Hz & Bandwidth of each received signal\\
StartUTC & 1 & String & & Nanosecond-resolution start time of pass (format: \mbox{YYYYMMDDThhmmss.sssssssssZ} according to ISO 8601) \\
SampInterval & 1 & Int32 & ns & Sampling interval of voltage signal\\
OutInterval & 1 & Int32 & ns & Measurement interval of signal-to-noise ratio, frequency values, their deviation, error and lock. Must be an integral multiple of SampInterval.\\
S4Interval & 1 & Int32 & ns & Measurement interval of $S_4$. Must be an integral multiple of SampInterval.\\
SigmaPhiInterval & 1 & Int32 & ns & Measurement interval of $\sigma _\phi$. Must be an integral multiple of SampInterval.\\
TECInterval & 1 & Int32 & ns & Measurement interval of TEC. Must be an integral multiple of SampInterval.\\
TLE & 3 & String & & Name of satellite and both lines of the TLE used to find its ephemeris\\
\hline
\end{tabular}
\end{scriptsize}

\paragraph{Data}- mid-level voltage, frequency and SNR data for the acquired signal on each frequency. \\
\begin{scriptsize}
\begin{tabular}{|l|l|l|c|p{7cm}|}
\hline
\textbf{Name} & \textbf{Size} & \textbf{Type} & \textbf{Units} & \textbf{Description} \\
\hline

Voltage & $(N_{freq}) \times (N_{samp})$ & Complex128 & V & In-phase and quadrature components of voltage for each frequency. If a sample is unavailable at a given time, it is denoted as NaN.\\
SNR & $(N_{freq}) \times (N_{out})$ & Float64 & dB & Signal-to-noise ratio for each frequency. Denoted as NaN if unavailable.\\
Frequency & $(N_{freq}) \times (N_{out})$ & Float64 & Hz & Actual frequency at which the signal was found at. Denoted as NaN if unavailabe.\\
Lock & $(N_{freq}) \times (N_{out})$ & Bool & T/F & Each sample value for each frequency will have a value that will be a true or false indicating if the signal had lock. If data
frequency data is unavailable then the lock value will default to false.\\
Deviation & $(N_{freq}) \times (N_{out})$ & Float64 & Hz & The deviation from the detected frequency and the frequency from the ephem calculator. If it is too high, lock is lost.
Denoted as NaN if unavailable.\\
Error & $(N_{freq}) \times (N_{out})$ & Float64 & Hz & The error of the frequency due to error propagation in the function. Error = $\frac{dy}{dx}\Delta{x}$. Denoted by NaN if
unavailable.\\
\hline
\end{tabular}
\end{scriptsize}

\paragraph{HighLevelData}- higher-level data processed from the complex voltages\\
\begin{scriptsize}
\begin{tabular}{|l|l|l|c|p{7cm}|}
\hline
\textbf{Name} & \textbf{Size} & \textbf{Type} & \textbf{Units} & \textbf{Description} \\
\hline

S4 & $(N_{freq}) \times (N_{s4}) \times 2$ & Float64 & & $S_4$ value over time for each frequency and its associated standard error. Denoted as NaN if unavailabe.\\
SigmaPhi & $(N_{freq}) \times (N_{\sigma \phi}) \times 2$ & Float64 & rad & $\sigma _\phi$ value over time for each frequency and associated standard error. Denoted as NaN if unavailable.\\
SlantTEC & $(N_{tec}) \times 4 \times 2$ & Float64 & e$^{-}$ / m$^{2}$ & Relative slant-path TEC values and errors for VHF/UHF/L-band, VHF/UHF, VHF/L-band and UHF/L-band in that order. NaN if unavailable.\\
VerticalTEC & $(N_{tec}) \times 4 \times 2$ & Float64 & TECU & Relative vertical TEC values and errors for VHF/UHF/L-band, VHF/UHF, VHF/L-band and UHF/L-band in that order. NaN if unavailable.\\

\hline
\end{tabular}
\end{scriptsize}

\paragraph{Ephemeris}- ephemeris data calculated from the TLE in the file\\
\begin{scriptsize}
\begin{tabular}{|l|l|l|c|p{7cm}|}
\hline
\textbf{Name} & \textbf{Size} & \textbf{Type} & \textbf{Units} & \textbf{Description} \\
\hline

Azimuth & $(N_{samp})$ & Float64 & deg & Azimuth of the satellite at a given time\\
Elevation & $(N_{samp})$ & Float64 & deg & Elevation of the satellite at a given time\\
Altitude & $(N_{samp})$ & Float64 & m & Sea-level altitude of the satellite at a given time\\
Range & $(N_{samp})$ & Float64 & m & Range of satellite to observer at a given time\\
Latitude & $(N_{samp})$ & Float64 & deg & Latitude of the satellite at a given time\\
Longitude & $(N_{samp})$ & Float64 & deg & Longitude of the satellite at a given time\\
RelativeVel & $(N_{samp})$ & Float64 & m/s & Relative radial velocity of satellite to observer at a given time\\
\hline
\end{tabular}
\end{scriptsize}

\paragraph{ReceiverConfig}- configuration of receiver for the pass\\
\begin{scriptsize}
\begin{tabular}{|l|l|l|c|p{7cm}|}
\hline
\textbf{Name} & \textbf{Size} & \textbf{Type} & \textbf{Units} & \textbf{Description} \\
\hline
SpectralInversion & 1 & bool && True if the spectrum is inverted, False if it is\\
AntennaDescription & 1 & String & & Description of antenna used\\
Latitude & 1 & Float64 & deg & Latitude of receiver\\
Longitude & 1 & Float64 & deg & Longitude of receiver\\
Altitude & 1 & Float64 & m & Sea-level altitude of receiver\\
Temperature & 1 & Float64 & celcius & The temperature at the receiver location\\
Pressure & 1 & Float64 & millibar & The pressure at the receiver location\\
Azimuth & 1 & Float64 & deg & Azimuth of direction in which the antenna points\\
Elevation & 1 & Float64 & deg & Elevation of direction in which the antenna points\\
\hline
\end{tabular}
\end{scriptsize}

\paragraph{} Additionally, the root directory contains a table named \textbf{SoftwareInfo} that details the elements in the processing chain used to create and process the data for the pass.\\
\begin{scriptsize}
\begin{tabular}{|l|l|p{7cm}|}
\hline
\textbf{Row Name} & \textbf{Type} & \textbf{Description} \\
\hline
SoftwareNumber & Int16 & Number designating the position of the software in the chain; a larger number indicates a later position\\
SoftwareName & String 1000 & Name describing the software\\
RevisionTag & String 1000 & Revision tag of software\\
ConfigFile & String 10000 & Configuration file used for the software\\
\hline
\end{tabular}
\end{scriptsize}
\end{small}
\end{document}
